\documentclass{hcij}

\begin{document}

\articletitle{Corporate Hackathons, How and Why? A Multiple Case Study of Motivation, Projects Proposal and Selection, Goal Setting, Coordination, and Outcomes}

\authorlist{Ei Pa Pa Pe-Than\textsuperscript{1}, Alexander Nolte\textsuperscript{2}, Anna Filippova\textsuperscript{3}, Christian Bird\textsuperscript{4}, Steve Scallen\textsuperscript{5}, James D. Herbsleb\textsuperscript{1}}

\affiliationlist{\textsuperscript{1}Institute for Software Research, School of Computer Science, Carnegie Mellon University, Pittsburgh, Pennsylvania, USA,\textsuperscript{2}Institute for Computer Science, University of Tartu, Tartu, Estonia, \textsuperscript{3}GitHub Inc., San Francisco, California, USA, \textsuperscript{4}Empirical Software Engineering, Microsoft Research, Redmond, Washington, USA, \textsuperscript{5}Microsoft Garage, Redmond, Washington, USA}

\titleauthorminibios

\miniauthorbio{Ei Pa Pa Pe-Than}{eipa@cmu.edu}{https://eipapa.github.io}{is a \emph{researcher} with an interest in \emph{collaboration and coordination in software development work}, especially to configure the future of collaborative work. She is a \emph{postdoctoral associate} in the \emph{Institute for Software Research} of the \emph{School of Computer Science} at \emph{Carnegie Mellon University}. She has a PhD in information science and a MSc in information systems from the Nanyang Technological University in Singapore, and a BSc (Hons) in computer science from the University of Computer Studies, Yangon in Myanmar.}

\miniauthorbio{Alexander Nolte}{alexander.nolte@ut.ee}{https://alexandernolte.github.io/}{is a lecturer at the Institute of Computer Science at the University of Tartu and an adjunct assistant professor at the Institute for Software Research at Carnegie Mellon University. His research interests include understanding how individuals collaborate and supporting them by designing and evaluating sociotechnical approaches that spark creativity, foster innovation, and improve collaboration. Nolte received a Ph.D. in information systems from the University of Duisburg-Essen in Germany. Contact him at alexander.nolte@ut.ee.}

\miniauthorbio{Anna Filippova}{annafil@gmail.com}{https://www.linkedin.com/in/annafilippova/}{is a \emph{senior manager} in the Data Science department at GitHub. Her research interests include fields of communications, organizational behavior, social psychology, and information systems, using both qualitative and quantitative methods. Filippova received a Ph.D. in communication and new media from the National University of Singapore.}

\miniauthorbio{Christian Bird}{email}{url}{is a \emph{principal researcher in Empirical Software Engineering} at \emph{Microsoft Research}. His research interests include empirical software engineering and focusing on ways to use data to guide decisions of stakeholders in large software projects. He has a PhD in computer science from the \emph{University of California, Davis}.}

\miniauthorbio{Steve Scallen}{sscallen@microsoft.com}{https://www.linkedin.com/in/steve-scallen-2221893/}{is a \emph{principal design researcher} at \emph{Microsoft Garage}. His research interests include the development of tools and resources for hackathons, hackers, hack teams, and hack projects. Scallen has a PhD from the \emph{University of Minnesota}.}

\miniauthorbio{James D. Herbsleb}{herbsleb@cmu.edu}{https://herbsleb.org/}{is the \emph{Director of the Institute for Software Research and Professor} in the \emph{School of Computer Science} at \emph{Carnegie Mellon University}. His research interests include the many ways in which software development work is organized and coordinated, including geographically distributed teams, open source ecosystems, and hackathons. He has a PhD in psychology and a JD in law from the \emph{University of Nebraska - Lincoln}, and a MS in computer science from the \emph{University of Michigan}.}

\titlenotes

\titlebackground{This article is based on a mixed-methods, multiple case study of five teams that participated in a large scale corporate hackathon in which two teams were “preexisting” teams (PETs) and three were newly-formed “flash” teams (FTs). This study provides a comprehensive understanding of team processes, conditions under which the projects were sustainable, and a broad array of outcomes of a corporate hackathon.}
\titleacknowledgments{We would like to thank all of our study's participants, the reviewers and editors, and our funders.}
\titlefunding{Funding for this work comes from a grant of the Alfred P. Sloan Foundation, “Enhancing Scientific Software Sustainability Through Community Code Engagements” and the National Science Foundation (NSF) (grant number 1901311).}
\titlesupplementarydata{The interview protocol can be found at https://eipapa.github.io/files/hcij-hackathon-interview-protocol.pdf.}

\titlehcieditorialrecord{First received on \emph{date}. Revisions received on \emph{date}, \emph{date}, and \emph{date}. Accepted by \emph{action-editor-name}. Final manuscript received on \emph{date}.}

\end{document}
