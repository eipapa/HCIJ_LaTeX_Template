\documentclass{hcij}
\bibliography{./hcij_example}

\begin{document}
\runninghead{Corporate Hackathons, How and Why?}
% Change this to your title
\articletitle{Corporate Hackathons, How and Why? A Multiple Case Study of Motivation, Projects Proposal and Selection, Goal Setting, Coordination, and Outcomes}

\authorlist{Ei Pa Pa Pe-Than\textsuperscript{1}, Alexander Nolte\textsuperscript{2}, Anna Filippova\textsuperscript{3}, Christian Bird\textsuperscript{4}, Steve Scallen\textsuperscript{5}, and James D. Herbsleb\textsuperscript{1}}

\affiliationlist{\textsuperscript{1}Institute for Software Research, School of Computer Science, Carnegie Mellon University, Pittsburgh, Pennsylvania, USA,\textsuperscript{2}Institute for Computer Science, University of Tartu, Tartu, Estonia, \textsuperscript{3}GitHub Inc., San Francisco, California, USA, \textsuperscript{4}Empirical Software Engineering, Microsoft Research, Redmond, Washington, USA, \textsuperscript{5}Microsoft Garage, Redmond, Washington, USA}

% Change this to your abstract text
\articleabstract{Time-bounded events such as hackathons, data dives, codefests, hack-days, sprints or edit-a-thons have increasingly gained attention from practitioners and researchers. Yet there is a paucity of research on corporate hackathons, which are nearly ubiquitous and present significant organizational, cultural, and managerial challenges. To provide a comprehensive understanding of team processes and broad array of outcomes of corporate hackathons, we conducted a mixed-methods, multiple case study of five teams that participated in a large scale corporate hackathon. Two teams were “preexisting” teams (PETs) and three were newly-formed “flash” teams (FTs). Our analysis revealed that PETs coordinated almost as if it was just another day at the office while creating innovations within the boundary of their regular work, whereas FTs adopted role-based coordination adapted to the hackathon context while creating innovations beyond the boundary of their regular work. Project sustainability depended on how much effort the team put into finding a home for their projects and whether their project was a good fit with existing products in the organization’s product portfolio. Moreover, hackathon participation had perceived positive effects on participants’ skills, careers, and social networks.}

\newpage

\thickhrrule

\tableofcontents

\thickhrrule

% Article body starts after this command
\articlebodystart

% Example table figure
\begin{figure}[b]
\caption{This is a table figure.}
\begin{tabular}{cccc}
\toprule
Col title 1 & Col title 2 & Col title 3 & Col title 4 \\
\midrule 
data 1 & data 2 & data 3 & data 4 \\
data 1 & data 2 & data 3 & data 4 \\
data 1 & data 2 & data 3 & data 4 \\
data 1 & data 2 & data 3 & data 4 \\
\bottomrule
\end{tabular}
\label{fig:testTablefigure}
\end{figure}

\section{Introduction}
Body text

\subsection{Level-2 heading}
Body text

\subsubsectionnonum{Level-3 Heading}
Body text

\paragraph{Level-4 Heading}
This is an example citation \cite{Doe2014}. This is another citation with many authors \cite{Davis2007}.

\subparagraph{Level-5 heading}
Body text

\section{Level-1 heading}
Body text

\subsection{Level-2 heading}
Body text

\subsubsectionnonum{Level-3 Heading}
Body text

\paragraph{Level-4 Heading}
Body text

\subparagraph{Level-5 heading}
Body text

% Example figure
\begin{figure}[t]
\caption{caption text}
\includegraphics[width=2in]{exampleImage.png}
\label{fig:testfigure}
\end{figure}

\articlebodyend

\articlenotes{Notes text}

% References
\begingroup
\raggedright
\titleformat*{\section}{\bfseries\Large\centering\MakeUppercase}
\printbibliography
\endgroup

% Appendices
\begin{appendices}
\appendixsection{Level-1 heading}

\appendixsubsection{Level-2 heading}
Body text

\appendixsubsection{Level-2 heading}
Body text

\appendixsubsubsection{Level-3 heading}
Body text

\appendixsection{Level-1 heading}

\appendixsubsection{Level-2 heading}
Body text

\end{appendices}

\end{document}
