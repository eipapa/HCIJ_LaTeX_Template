\documentclass{hcij}
\usepackage[utf8]{inputenc}
\bibliography{./hcij_example}

\begin{document}
\title{Corporate Hackathons, How and Why? A Multiple Case Study of Motivation, Projects Proposal and Selection, Goal Setting, Coordination, and Outcomes}

\authorlist{Ei Pa Pa Pe-Than\textsuperscript{1}, Alexander Nolte\textsuperscript{2}, Anna Filippova\textsuperscript{3}, Christian Bird\textsuperscript{4}, Steve Scallen\textsuperscript{5}, and James D. Herbsleb\textsuperscript{1}}

\affiliationlist{\textsuperscript{1}Institute for Software Research, School of Computer Science, Carnegie Mellon University, Pittsburgh, Pennsylvania, USA,\textsuperscript{2}Institute for Computer Science, University of Tartu, Tartu, Estonia, \textsuperscript{3}GitHub Inc., San Francisco, California, USA, \textsuperscript{4}Empirical Software Engineering, Microsoft Research, Redmond, Washington, USA, \textsuperscript{5}Microsoft Garage, Redmond, Washington, USA}
\date{April 2020}

\articleabstract{Time-bounded events such as hackathons, data dives, codefests, hack-days, sprints or edit-a-thons have increasingly gained attention from practitioners and researchers. Yet there is a paucity of research on corporate hackathons, which are nearly ubiquitous and present significant organizational, cultural, and managerial challenges. To provide a comprehensive understanding of team processes and broad array of outcomes of corporate hackathons, we conducted a mixed-methods, multiple case study of five teams that participated in a large scale corporate hackathon. Two teams were “preexisting” teams (PETs) and three were newly-formed “flash” teams (FTs). Our analysis revealed that PETs coordinated almost as if it was just another day at the office while creating innovations within the boundary of their regular work, whereas FTs adopted role-based coordination adapted to the hackathon context while creating innovations beyond the boundary of their regular work. Project sustainability depended on how much effort the team put into finding a home for their projects and whether their project was a good fit with existing products in the organization’s product portfolio. Moreover, hackathon participation had perceived positive effects on participants’ skills, careers, and social networks.}

\newpage

\articlebodystart

\section{Introduction}
In recent years, time-bounded events such as hackathons, data dives, codefests, hack-days, sprints, or edit-a-thons have experienced a steep increase in popularity. During these and similar events people form teams – often ad hoc – and engage in intense collaboration over a short period of time. Collegiate events that are organized by the largest hackathon league alone attract over 65,000 participants among more than 200 events each year (e.g., https://mlh.io/about). But it is not collegiate events alone. Hackathons have become a global phenomenon (Taylor & Clarke, 2018) covering a variety of contexts ranging from corporations (Rosell, Kumar, & Shepherd, 2014; Frey & Luks, 2016) to higher education (Kienzler & Fontanesi, 2017) and civic engagement (Baccarne, Van Compernolle, & Mechant, 2015; Henderson, 2015; Hartmann, Mainka, & Stock, 2019).
These events vary along several dimensions: whether teams know each other beforehand (Möller et al., 2014), whether the event is structured as a competition with prizes, whether the event is open only to members of a single organization, whether the participants are students or the public addressing civic issues (Carruthers, 2014; Porter, Bopp, Gerber, & Voida, 2017; Taylor & Clarke, 2018), and whether the desired outcome is primarily a product innovation (Henderson, 2015;  Rosell et al., 2014), learning a new skill (Decker, Eiselt, & Voll, 2015; Fowler, 2016; Lara & Lockwood, 2016; Nandi & Mandernach, 2016), forming a community around a cause (Möller et al., 2014), advancing a technical work that requires intensive focus by a group (Anonymous, 2019), or just having fun.
Corporate hackathons are a special kind of time-bounded event and often aim at broadening participation in the corporate innovation network, in much the same spirit as IBM’s “Innovation Jam” (Bjelland & Wood, 2008; Rosell et al., 2014; Gibson, Hardy, & Ronald Buckley, 2014) (cf. section 2.1). The participants have their own goals for participation, such as learning (Nandi & Mandernach, 2016) and networking (Möller et al., 2014), which may or may not be the same as the organizers’ goals. Although there has been a growing body of work around hackathons, very few studies to date have examined corporate hackathons, with a few notable exceptions (Nolte, Pe-Than, Filippova, Bird, Scallen, & Herbsleb, 2018; Pe-Than, Nolte, Filippova, Bird, Scallen, & Herbsleb, 2018; Komssi, Pichlis, Raatikainen, Kindström, & Järvinen, 2015). Some of them examined the potential outcomes of corporate hackathons such as project sustainability (Nolte et al., 2018; Komssi et al., 2015) while other presented different ways that corporate hackathons could be designed (Pe-Than et al., 2018). Yet it is unclear how different outcomes are achieved and what conditions favored achieving various outcomes.
Prior work on traditional teams regards team familiarity as an important dimension which consistently found to influence team coordination (Espinosa, Slaughter, Kraut, & Herbsleb, 2007; Hinds & Cramton, 2014). As hackathons afford several ways of organizing teams, including both self-selected and moderated teams (Trainer, Kalyanasundaram, Chaihirunkarn, & Herbsleb, 2016), teams may consist of members with varying levels of familiarity. Although coordination seems straightforward for pre-existing teams, collaboration in teams with members who had not worked together before is challenging due to the lack of existing knowledge about their team members (Goodman & Leyden, 1991). The hackathon setting amplifies such challenges for newly formed teams as they need to work together effectively in an extremely compressed time scale, typically 2-5 days, with no or very little time to get familiarized with other members. Prior work also suggests that working under such intense time pressure requires teams to set realistic expectations and goals (Henderson, 2015; Porter et al., 2017), which is more important for ad hoc teams with members who otherwise would not work together (Komssi et al., 2015; Möller et al., 2014). Hence, hackathon teams are faced with a major coordination challenge, particularly for teams whose members are not familiar with each other (or teams formed just for the hackathon or “flash” teams (FTs)). In contrast, teams with members who have worked together before (or pre-existing teams (PETs)) possess existing knowledge about their members that they may leverage to coordinate.
Corporate hackathons differ from the sorts of hackathons usually studied in the literature, which typically exist outside any stable organizational context and bring together people who generally have not worked together – or even met each other – before. In corporate hackathons, participants may know each other well, or be relative strangers, but they share a corporate culture and overall purpose. This can be expected to influence how they work, and the opportunities that arise for continuing a project after the event. Their enduring role as corporate employees can also be expected help shape their perceptions of the value they place on the different influences the event had on them.
This paper aims to address the following research questions in the context of a corporate hackathon: what were the team processes, and how did they differ between PETs and FTs? (RQ1), what were the conditions that contributed to sustaining the projects after the event? (RQ2), and what impacts did participants believe the event had on them? (RQ3). To address this gap, we conducted a mixed-methods, multiple case study of five teams that participated in the 2017 Microsoft OneWeek Hackathon. The OneWeek Hackathon is one of the largest corporate hackathons in the world with more than 18,000 employees working on more than 4,700 projects worldwide. We focused on the largest site where we selected five teams based on team size, familiarity among team members, and the relationship between their hackathon project and their everyday work. Three of the teams we selected are FTs, and two are PETs. We observed the selected teams for the entire course of the hackathon, and conducted interviews of team members before, immediately after, and four months after the hackathon.
Our paper contributes several new insights about how hackathons operate in a corporate context. Our analysis revealed that FTs and PETs adopted different styles of coordination, with PET’s working in their accustomed style while FTs adopted a role-based coordination strategy, based on standard corporate roles, but modified to fit the hackathon context and often executed uncertainly, as many participants were trying out roles they had not experienced before. PETs and FTs also worked on different types of projects, with PETs developing nearly fully functional products immediately useful for their current works situation, while FTs tended toward lightly-engineered prototypes while aiming their efforts at a wider audience, as they sought a permanent home for their project. The sustainability of projects after the hackathon, for both FTs and PETs, was enhanced when members took on tasks where they were already highly skilled, and focused their learning efforts on the specific skills needed for their project. Sustainability was also enhanced if leaders were career-oriented, and focused on meticulous preparation and execution of their project rather than learning or innovation outside their areas of expertise. Our results also suggest that the perceived benefits to FT members from participation in the hackathon did not all derive from the project itself, and the attention it received, but also involved skill development, networking, trying on new roles, and greater organizational knowledge. The benefits to PETs, on the other hand, mainly derived from demonstrating skills and creating something of value within their existing chain of command.

\subsection{Background}
\end{document}
